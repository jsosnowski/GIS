\section{Implementacja}

Zgodnie z założeniami projektu, algorytm Fleury'ego będzie dostarczony w postaci dwóch aplikacji stworzonych w językach Java i C++. 
Aplikacja testująca wydajność będzie odrębnym modułem.

Z różnych reprezentacji grafów dostępnych w książce \cite{Cormen} został wybrany format wierzchołkowy.
Struktura grafu będzie więc reprezentowana za pomocą list wierzchołków powiązanych krawędziami, a nie listy krawędzi (jak w konkurencyjnej reprezentacji). 
Z tego powodu, etykietowane będą wyłącznie wierzchołki.
Krawędzie są rozróżniane na podstawie wierzchołków z którymi są incydentne.
W przypadku grafów skierowanych, kolejność wierzchołków wyznacza zwrot krawędzi.
Krawędź skierowana \textit{(u, v)} jest skierowana od wierzchołka \textit{u} do \textit{v}.
Takie ustalenia wpływają na format danych wejściowych -- zaprezentowany na tablicy \ref{FormatPliku}. Dane wejściowe mają postać plików.

\subsection{Format danych wejściowych -- grafów}

\begin{table}[b]
\centering
\begin{tabular}{l|r}

\begin{tabular}{@{}l@{}}
~~~~~~~~~~ \textit{Plik A} ~~~~~~~~~~ \\
 \\
undirected 4\\
0 : 1 2 3\\
1 : 0 2 3\\
2 : 0 1 3\\
3 : 0 1 2\\

\end{tabular}
&
\begin{tabular}{@{}l@{}}
~~~~~~~~~~ \textit{Plik B} ~~~~~~~~~~ \\
\\
directed 4\\
0 : 1 3\\
1 : 2 0\\
2 : 3 1\\
3 : 0 2

\end{tabular}
\end{tabular}
\caption{Format pliku przechowujący graf.}
\label{FormatPliku}
\end{table}


Pierwsza linia ma wyłącznie charakter specyfikacji i stanowi informacje nadmiarowe.
Zawiera pojedyncze słowo informujące czy graf przechowywany w pliku jest skierowany (ang. \textit{directed}) lub nieskierowany (ang. \textit{undirected}). 
Dalej pojawia się liczba wierzchołków w grafie.

Kolejne linie są już bezpośrednią reprezentacją grafu.
Każda jest rozdzielona znakiem dwukropka na dwie części: pierwsza specyfikuje nazwę wierzchołka, druga jest listą wierzchołków z którymi jest połączony dany wierzchołek.
W ramach tego projektu nazwy będą liczbami dziesiętnymi.

Obie informacje z linii specyfikacji (pierwszej linii) można wywnioskować na podstawie analizy reszty pliku.
Liczba wierzchołków to po prostu liczba linii zawierających opis grafu.
Natomiast typ grafu można wydedukować zauważając, że graf nieskierowany, który zawiera krawędź (u, v) musi również zawierać krawędź (v, u).
Mimo to linia specyfikacji znacząco ułatwia etap analizy syntaktycznej pliku wejściowego.

