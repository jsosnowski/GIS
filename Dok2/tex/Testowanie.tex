\section{Testowanie}

Działanie aplikacji zostanie sprawdzone na podstawie dwóch zestawów danych.
Pierwszy jest zestawem przeznaczonym dla testów akceptacyjnych.
Zadaniem drugiego będzie sprawdzenie wydajności aplikacji. 

\subsection{Projekt testów akceptacyjnych}

Zestaw tych testów składa się z kilkunastu niewielkich grafów w dwóch wariantach.
Pierwszy wariant to grafy dla których analitycznie udowodniono istnienie cykli Eulera (spełniają warunek konieczny). Przedstawia je tablica \ref{GrafyZCyklemEulera}.
Natomiast drugi wariant składa się z grafów, które nie mogą takiego cyklu zawierać.
Odpowiadają one pierwszemu wariantowi z losową modyfikacją jednej krawędzi grafu z zastrzeżeniem zachowania spójności. Dodanie krawędzi bądź usunięcie zaburza warunek konieczny istnienia cyklu Eulera.

\subsection{Generacja testów wydajnościowych}

Zestaw danych dla testów wydajnościowych będzie pochodzić z generatora grafów eulerowskich.
Sam generator też jest elementem projektu. 
Jego działanie opierać się będzie na tworzeniu grafów o zadanej liczbie wierzchołków poprzez losowe rozmieszczenie krawędzi pomiędzy nimi.
Liczbę krawędzi kontroluje parametr wypełnienia (gęstości), który jest równy proporcji liczby krawędzi w grafie do potencjalnej liczby wszystkich możliwych krawędzi. I tak dla grafu nieskierowanego jest to:

\[ 
wypelnienie = \frac{liczbaKrawedzi}{\frac{N(N-1)}{2}}
\]
dla grafu skierowanego:
\[ 
wypelnienie = \frac{liczbaKrawedzi}{N(N-1)}
\]
gdzie, N to liczba wierzchołków w grafie.

Dzięki zdefiniowaniu parametru wypełnienia, nie ma potrzeby podawania konkretnej liczby krawędzi podczas generowania nowego grafu. Przykładowo wystarczy informacja, że powinien on być wypełniony co najmniej w 50\%.

Projekt testów zakłada utworzenie zestawu grafów o liczbie wierzchołków wahającej się od 5 do około 10000.
Wypełnienie będzie zmienne. Ale raczej ze wskazaniem na struktury bardziej zbliżone liczbą krawędzi do grafów pełnych.
Powodem tego jest fakt, iż grafy o niewielkim wypełnieniu nie stanowią wyzwania dla omawianego algorytmu, którego złożoność jest wprost proporcjonalna do liczby krawędzi.

Każdy test będzie powtarzany ustaloną liczbę razy w celu wyznaczenia parametrów statystycznych czasu wykonania (między innymi: czas średni, mediana, odchylenie standardowe).



\begin{table}
\caption{Grafy zawierające cykl Eulera -- stanowią podstawę testów akceptacyjnych.}
\label{GrafyZCyklemEulera}
\flushleft

\setlength\tabcolsep{2pt}
\begin{tabular}{@{}ccc@{}}

% Trójkąt nieskierowany
%\fbox{
\begin{tikzpicture}[-,>=stealth',shorten >=1pt,auto,node distance=1.8cm,
                    semithick]
  \tikzstyle{every state}=[fill=red,draw=none,text=white]

  \node[state] 		   (A)                    {$0$};
  \node[state]         (B) [below right of=A] {$1$};
  \node[state]         (C) [below left of=A] {$2$};

  \path (A) edge node {} (B)
        (B) edge node {} (C)
        (C) edge node {} (A)
  ; %end path

\end{tikzpicture} %}//end of \fbox - dodaje ramkę
&
%Trójkąt skierowany
\begin{tikzpicture}[->,>=stealth',shorten >=1pt,auto,node distance=1.8cm,
                    semithick]
  \tikzstyle{every state}=[fill=red,draw=none,text=white]

  \node[state] 		   (A)                    {$0$};
  \node[state]         (B) [below right of=A] {$1$};
  \node[state]         (C) [below left of=A] {$2$};

  \path (A) edge node {} (B)
        (B) edge node {} (C)
        (C) edge node {} (A)
  ; %end path

\end{tikzpicture} 
&
% Prostokąt nieskierowany
\begin{tikzpicture}[-,>=stealth',shorten >=1pt,auto,node distance=1.8cm,
                    semithick]
  \tikzstyle{every state}=[fill=red,draw=none,text=white]

  \node[state] 		   (A)                    {$0$};
  \node[state]         (B) [right of=A] {$1$};
  \node[state]         (C) [below of=B] {$2$};
  \node[state]         (D) [below of=A] {$3$};

  \path (A) edge node {} (B)
        (B) edge node {} (C)
        (C) edge node {} (D)
        (D) edge node {} (A)
  ; %end path

\end{tikzpicture}
\\
% Prostokąt skierowany
\begin{tikzpicture}[->,>=stealth',shorten >=1pt,auto,node distance=1.8cm,
                    semithick]
  \tikzstyle{every state}=[fill=red,draw=none,text=white]

  \node[state] 		   (A)                    {$0$};
  \node[state]         (B) [right of=A] {$1$};
  \node[state]         (C) [below of=B] {$2$};
  \node[state]         (D) [below of=A] {$3$};

  \path (A) edge node {} (B)
        (B) edge node {} (C)
        (C) edge node {} (D)
        (D) edge node {} (A)
  ; %end path

\end{tikzpicture}
&
% Prostokąt skierowany 2
\begin{tikzpicture}[->,>=stealth',shorten >=1pt,auto,node distance=1.8cm,
                    semithick]
  \tikzstyle{every state}=[fill=red,draw=none,text=white]

  \node[state] 		   (0)                    {$0$};
  \node[state]         (1) [right of=0] {$1$};
  \node[state]         (2) [below of=1] {$2$};
  \node[state]         (3) [below of=0] {$3$};

  \path (0) edge node {} (2)
        (1) edge node {} (0)
        (2) edge node {} (3)
        (3) edge node {} (1)
  ; %end path

\end{tikzpicture}
&
% Pięciokąt 1
\begin{tikzpicture}[->,>=stealth',shorten >=1pt,auto,node distance=1.8cm,
                    semithick]
  \tikzstyle{every state}=[fill=red,draw=none,text=white]

  \node[state] 		   (0)                    {$0$};
  \node[state]         (1) [below right of=0] {$1$};
  \node[state]         (2) [below of=1] {$2$};
  \node[state]         (3) [left of=2] {$3$};
  \node[state]         (4) [below left of=0] {$4$};

  \path (0) edge node {} (1)
        (1) edge node {} (2)
        (2) edge node {} (3)
        (3) edge node {} (4)
        (4) edge node {} (0)
  ; %end path

\end{tikzpicture}
\\
% Pięciokąt 2
\begin{tikzpicture}[->,>=stealth',shorten >=1pt,auto,node distance=1.8cm,
                    semithick]
  \tikzstyle{every state}=[fill=red,draw=none,text=white]

  \node[state] 		   (0)                    {$0$};
  \node[state]         (1) [below right of=0] {$1$};
  \node[state]         (2) [below of=1] {$2$};
  \node[state]         (3) [left of=2] {$3$};
  \node[state]         (4) [below left of=0] {$4$};

  \path (0) edge node {} (1)
        (0) edge node {} (2)
        (1) edge node {} (4)
        (2) edge node {} (3)
        (3) edge node {} (0)
        (4) edge node {} (0)
  ; %end path

\end{tikzpicture}
&
% Pięciokąt 3
\begin{tikzpicture}[->,>=stealth',shorten >=1pt,auto,node distance=1.8cm,
                    semithick]
  \tikzstyle{every state}=[fill=red,draw=none,text=white]

  \node[state] 		   (0)                    {$0$};
  \node[state]         (1) [below right of=0] {$1$};
  \node[state]         (2) [below of=1] {$2$};
  \node[state]         (3) [left of=2] {$3$};
  \node[state]         (4) [below left of=0] {$4$};

  \path (0) edge node {} (1)
        (1) edge node {} (3)
        (2) edge node {} (0)
        (3) edge node {} (4)
        (4) edge node {} (2)
  ; %end path

\end{tikzpicture}
&

% Pięciokąt 4
\begin{tikzpicture}[->,>=stealth',shorten >=1pt,auto,node distance=1.8cm,
                    semithick]
  \tikzstyle{every state}=[fill=red,draw=none,text=white]

  \node[state] 		   (0)                    {$0$};
  \node[state]         (1) [below right of=0] {$1$};
  \node[state]         (2) [below of=1] {$2$};
  \node[state]         (3) [left of=2] {$3$};
  \node[state]         (4) [below left of=0] {$4$};

  \path (0) edge node {} (4)
        (1) edge node {} (2)
        (2) edge node {} (3)
        (3) edge node {} (0)
        (4) edge node {} (1)
  ; %end path

\end{tikzpicture}
\\
% Pięciokąt 5
\begin{tikzpicture}[->,>=stealth',shorten >=1pt,auto,node distance=1.8cm,
                    semithick]
  \tikzstyle{every state}=[fill=red,draw=none,text=white]

  \node[state] 		   (0)                    {$0$};
  \node[state]         (1) [below right of=0] {$1$};
  \node[state]         (2) [below of=1] {$2$};
  \node[state]         (3) [left of=2] {$3$};
  \node[state]         (4) [below left of=0] {$4$};

  \path (0) edge node {} (1)
        (1) edge node {} (4)
        (2) edge node {} (3)
        (3) edge node {} (4)
        (4) edge node {} (0)
        (4) edge node {} (2)
  ; %end path

\end{tikzpicture}
&
% Pięciokąt 6
\begin{tikzpicture}[->,>=stealth',shorten >=1pt,auto,node distance=1.8cm,
                    semithick]
  \tikzstyle{every state}=[fill=red,draw=none,text=white]

  \node[state] 		   (0)                    {$0$};
  \node[state]         (1) [below right of=0] {$1$};
  \node[state]         (2) [below of=1] {$2$};
  \node[state]         (3) [left of=2] {$3$};
  \node[state]         (4) [below left of=0] {$4$};

  \path (0) edge node {} (2)
        (1) edge node {} (3)
        (2) edge node {} (4)
        (3) edge node {} (4)
        (4) edge node {} (0)
        (4) edge node {} (1)
  ; %end path

\end{tikzpicture}
&
% Pięciokąt 7
\begin{tikzpicture}[->,>=stealth',shorten >=1pt,auto,node distance=1.8cm,
                    semithick]
  \tikzstyle{every state}=[fill=red,draw=none,text=white]

  \node[state] 		   (0)                    {$0$};
  \node[state]         (1) [below right of=0] {$1$};
  \node[state]         (2) [below of=1] {$2$};
  \node[state]         (3) [left of=2] {$3$};
  \node[state]         (4) [below left of=0] {$4$};

  \path (0) edge node {} (1)
        (1) edge node {} (2)
        (1) edge node {} (3)
        (2) edge node {} (4)
        (3) edge node {} (4)
        (4) edge node {} (0)
        (4) edge node {} (1)
  ; %end path

\end{tikzpicture}
\\
% Pięciokąt 8
\begin{tikzpicture}[<->,>=stealth',shorten >=1pt,auto,node distance=1.8cm,
                    semithick]
  \tikzstyle{every state}=[fill=red,draw=none,text=white]

  \node[state] 		   (0)                    {$0$};
  \node[state]         (1) [below right of=0] {$1$};
  \node[state]         (2) [below of=1] {$2$};
  \node[state]         (3) [left of=2] {$3$};
  \node[state]         (4) [below left of=0] {$4$};

  \path (0) edge node {} (1)
        (0) edge node {} (2)
        (0) edge node {} (3)
        (0) edge node {} (4)
        (1) edge node {} (2)
        (1) edge node {} (3)
        (1) edge node {} (4)
        (2) edge node {} (3)
        (2) edge node {} (4)
        (3) edge node {} (4)
  ; %end path

\end{tikzpicture}
&
% Sześciokąt 1
\begin{tikzpicture}[->,>=stealth',shorten >=1pt,auto,node distance=1.8cm,
                    semithick]
  \tikzstyle{every state}=[fill=red,draw=none,text=white]

  \node[state] 		   (0)                    {$0$};
  \node[state]         (1) [right of=0] {$1$};
  \node[state]         (2) [below right of=1] {$2$};
  \node[state]         (3) [below left of=2] {$3$};
  \node[state]         (4) [left of=3] {$4$};
  \node[state]         (5) [below left of=0] {$5$};

  \path (0) edge node {} (3)
        (1) edge node {} (2)
        (2) edge node {} (5)
        (3) edge node {} (4)
        (4) edge node {} (1)
        (5) edge node {} (0)
  ; %end path

\end{tikzpicture}
&
% Sześciokąt 2
\begin{tikzpicture}[->,>=stealth',shorten >=1pt,auto,node distance=1.8cm,
                    semithick]
  \tikzstyle{every state}=[fill=red,draw=none,text=white]

  \node[state] 		   (0)                    {$0$};
  \node[state]         (1) [right of=0] {$1$};
  \node[state]         (2) [below right of=1] {$2$};
  \node[state]         (3) [below left of=2] {$3$};
  \node[state]         (4) [left of=3] {$4$};
  \node[state]         (5) [below left of=0] {$5$};

  \path (0) edge node {} (3)
        (1) edge node {} (2)
        (2) edge node {} (5)
        (3) edge node {} (1)
        (4) edge node {} (0)
        (5) edge node {} (4)
  ; %end path

\end{tikzpicture}


%
%\begin{tikzpicture}[->,>=stealth',shorten >=1pt,auto,node distance=1.8cm,
%                    semithick]
%  \tikzstyle{every state}=[fill=red,draw=none,text=white]
%
%  \node[initial,state] (A)                    {$q_a$};
%  \node[state]         (B) [above right of=A] {$q_b$};
%  \node[state]         (D) [below right of=A] {$q_d$};
%  \node[state]         (C) [below right of=B] {$q_c$};
%  \node[state]         (E) [below of=D]       {$q_e$};
%
%  \path (A) edge              node {0,1,L} (B)
%            edge              node {1,1,R} (C)
%        (B) edge [loop above] node {1,1,L} (B)
%            edge              node {0,1,L} (C)
%        (C) edge              node {0,1,L} (D)
%            edge [bend left]  node {1,0,R} (E)
%        (D) edge [loop below] node {1,1,R} (D)
%            edge              node {0,1,R} (A)
%        (E) edge [bend left]  node {1,0,R} (A);
%\end{tikzpicture}
%


\end{tabular}

\end{table}