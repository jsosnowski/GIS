\section{Podsumowanie}

W ramach tego projektu algorytm Fleury'ego został poprawnie zaimplementowany w dwóch najbardziej popularnych językach obecnego świata IT.
Przeprowadzono szereg różnorodnych wyszukiwań cykli Eulera na losowych grafach o różnych liczbach wierzchołków i krawędzi.
Testowano grafy od najmniejszego trój-wierzchołkowego, po takie o liczbie wierzchołków rzędu tysięcy.
Ponadto dokonano wielu badań, których wyniki zostały zaprezentowane w niniejszym sprawozdaniu.

Na podstawie zdobytych doświadczeń, a także w ramach podsumowania można wysnuć wniosek o istnieniu \textbf{optymalizacji} w kontekście zmiany języka implementacji dla algorytmów grafowych.
Innymi słowy, pokazano tutaj, że ten sam algorytm osiąga znacząco różne rezultaty w różnych językach.
Nie zakłada się, że cechą obu języków jest fakt, że jeden jest efektywniejszy od drugiego.
Ale posiadając kod Javy implementujący pewien algorytm grafowy (np. poszukujący cykli Eulera) jest wielce prawdopodobne, że jego ,,przepisanie'' w języku C++ przyniesie korzyść w postaci zmniejszonej konsumpcji czasu.
Jest to w rzeczywistości pewien rodzaj optymalizacji, który można osiągnąć stosunkowo niskim kosztem oraz bez ryzyka straty czytelności samego kodu źródłowego.

