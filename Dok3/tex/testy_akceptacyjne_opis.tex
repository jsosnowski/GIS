\subsection{Testy akceptacyjne}
\label{sub:testy_akceptacyjne}

Zanim oba rozwiązania zostały poddane ocenie wydajności, musiały przejść zestaw testów akceptacyjnych.
Mowa o poszukiwaniu cykli Eulera w prostych grafach wejściowych.
Kilka przykładów przedstawia tabela~\ref{Grafy3TestyAkceptacyjne}.
Więcej grafów znajduje się w katalogach z testami projektu oraz w poprzedniej dokumentacji.

Dla wyjaśnienia, należy zaznaczyć, że ostatni przykład z tabeli \ref{Grafy3TestyAkceptacyjne}
jest tym samym grafem co pierwszy, lecz z dodaną dodatkową krawędzią $(0,3)$.
Taka modyfikacja, sprawiła, że nie można znaleźć cyklu Eulera. 
Istotnie, idąc tą samą drogą, co dla pierwszego grafu, odwiedzone zostaną wszystkie krawędzie oprócz $(0,3)$.
%To pokazuje, że dodanie dowolnej pojedynczej krawędzi do grafu zawierającego cykl Eulera, prowadzi do struktury w której takiego cyklu znaleźć nie można.

Wynikowy cykl Eulera może być zwracany jako tekst przekazany na standardowe wyjście znakowe systemu operacyjnego lub zapisany do pliku.

\begin{table}
\caption{Przykłady grafów stanowiących część testów akceptacyjnych.}
\label{Grafy3TestyAkceptacyjne}
\center

\setlength\tabcolsep{2pt}
\begin{tabular}{l|c|l}

Plik wejściowy & Wygląd grafu & Wynik algorytmu \\
\hline
\hline

\begin{tabular}{l}
directed 5	\\
0 : 1		\\
1 : 4		\\
2 : 3		\\
3 : 4		\\	
4 : 0 2		\\
\end{tabular}
&
\begin{tabular}{l}
% Pięciokąt 5
\begin{tikzpicture}[->,>=stealth',shorten >=1pt,auto,node distance=1.8cm,
                    semithick]
  \tikzstyle{every state}=[fill=red,draw=none,text=white]

  \node[state] 		   (0)                    {$0$};
  \node[state]         (1) [below right of=0] {$1$};
  \node[state]         (2) [below of=1] {$2$};
  \node[state]         (3) [left of=2] {$3$};
  \node[state]         (4) [below left of=0] {$4$};

  \path (0) edge node {} (1)
        (1) edge node {} (4)
        (2) edge node {} (3)
        (3) edge node {} (4)
        (4) edge node {} (0)
        (4) edge node {} (2)
  ; %end path

\end{tikzpicture}
\end{tabular}
&
0, 1, 4, 2, 3, 4, 0
\\
\hline

\begin{tabular}{l}
directed 6	\\
0 : 3		\\
1 : 2		\\
2 : 5		\\
3 : 4		\\
4 : 1		\\
5 : 0		\\
\end{tabular}
&
\begin{tabular}{l}
% Sześciokąt 1
\begin{tikzpicture}[->,>=stealth',shorten >=1pt,auto,node distance=1.8cm,
                    semithick]
  \tikzstyle{every state}=[fill=red,draw=none,text=white]

  \node[state] 		   (0)                    {$0$};
  \node[state]         (1) [right of=0] {$1$};
  \node[state]         (2) [below right of=1] {$2$};
  \node[state]         (3) [below left of=2] {$3$};
  \node[state]         (4) [left of=3] {$4$};
  \node[state]         (5) [below left of=0] {$5$};

  \path (0) edge node {} (3)
        (1) edge node {} (2)
        (2) edge node {} (5)
        (3) edge node {} (4)
        (4) edge node {} (1)
        (5) edge node {} (0)
  ; %end path

\end{tikzpicture}
\end{tabular}
&
3, 4, 1, 2, 5, 0, 3
\\
\hline

\begin{tabular}{l}
directed 5	\\
\textbf{0} : 1 \textbf{3}		\\
1 : 4		\\
2 : 3		\\
3 : 4		\\
4 : 0 2		\\
\end{tabular}
&
\begin{tabular}{l}
% Pięciokąt 5 z krawedzia (0,3)
\begin{tikzpicture}[->,>=stealth',shorten >=1pt,auto,node distance=1.8cm,
                    semithick]
  \tikzstyle{every state}=[fill=red,draw=none,text=white]

  \node[state] 		   (0)                    {$0$};
  \node[state]         (1) [below right of=0] {$1$};
  \node[state]         (2) [below of=1] {$2$};
  \node[state]         (3) [left of=2] {$3$};
  \node[state]         (4) [below left of=0] {$4$};

  \path (0) edge node {} (1)
  		(0) edge node {} (3)
        (1) edge node {} (4)
        (2) edge node {} (3)
        (3) edge node {} (4)
        (4) edge node {} (0)
        (4) edge node {} (2)
  ; %end path

\end{tikzpicture}
\end{tabular}
&
Euler Path: (not found)
\\
\hline
\hline

\end{tabular}

\end{table}



