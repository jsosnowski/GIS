\section{Testowanie}

\subsection{Wprowadzenie}

Przedmiotem projektu był algorytm odnajdowania cyklu Eulera. 
Jako cel została przyjęta jego implementacja w językach C++ i Java.
Podwójna implementacja miała umożliwić eksperyment polegający na dokładnym zbadaniu wydajności obu podejść.
Ostatecznie w niniejszym sprawozdaniu zostanie uzasadniona odpowiedź na pytania:
\begin{quote}
\textit{
Czy możliwa jest optymalizacja algorytmu poprzez zmianę języka (środowiska) jego implementacji?
A jeśli tak, to jak bardzo znaczącą poprawę można uzyskać?
}
\end{quote}

\subsection{Warunki eksperymentów}

Wstępnie założono, że żadna z implementacji nie zostanie poddana optymalizacji kodu.
Cenniejsze jest uzyskanie czytelnego rozwiązania, aniżeli skomplikowanego i niezrozumiałego kodu.
Jest to sytuacja zbliżona do rzeczywistości -- gdzie zazwyczaj nie ma miejsca na wyścig optymalizacji.

Schemat działania jest wspólny dla obu podejść.
Po pierwsze należy wczytać dane wejściowe (graf) z pliku oraz zbudować odpowiednią strukturę danych w pamięci programu.
Po drugie należy uruchomić implementację algorytmu Fleury'ego (opisanego dokładnie w poprzedniej dokumentacji).


\subsection{Środowisko testowe}

Wszystkie testy były uruchamiane w systemie Ubuntu w wersji 14.04.
Pomiary wydajności zostały wykonane z wykorzystaniem środowiska języka R (za pomocą narzędzia R-Studio).
Znaczenie ma również konfiguracja sprzętowa, która w tym przypadku była oparta na dwurdzeniowym procesorze taktowanym zegarem 2,2 GHz.
Kod źródłowy został skompilowany przez kompilator GCC 4.8 w przypadku języka C++.
Natomiast aplikacja stworzona w Javie podczas testów była uruchamiana z wykorzystaniem platformy JRE w wersji  1.8.

\textbf{Pomiar czasu} był wykonywany z dokładnością do jednej mikrosekundy.
Lecz dla zachowania czytelności niniejszej dokumentacji oraz z uwagi na charakter prowadzonych testów pomiar (czas trwania dłuższy niż sekunda), wyniki są zapisywane z mniejszą precyzją.


